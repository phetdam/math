\documentclass{article}
\usepackage[margin=1in]{geometry}
\usepackage{amsmath, amssymb, amsfonts}
% change equation numbering to section.eq_num
\numberwithin{equation}{section}
% set paragraph indent
\setlength\parindent{0pt}
% makes clickable links to sections
\usepackage{hyperref}
% make the link colors blue, as well as cite colors
\hypersetup{
    colorlinks, linkcolor = blue, citecolor = blue, urlcolor = magenta
}

\title{Math problems for BAC returning members}
\author{
    Derek Huang\thanks{
        NYU Stern `2021, former \href{
            https://sternbac.org/insightteam.html
        }{BAC Advanced Team Research}.
    }
}
\date{January 9, 2021}

\begin{document}

\maketitle

\begin{abstract}
    A collection of problems from which some may be used to test returning
    members' conceptual understanding. Topics range from probability theory,
    linear algebra, convexity, and some model-related problems. For the more
    difficult problems, appropriate hints are provided as guides, as none of
    these problems are supposed to be too difficult by themselves. Rather, they
    are supposed to test members' application of fundamental mathematics to
    potentially novel situations, and thus each problem contains enough
    requisite background to be solved without specialized domain knowledge.
\end{abstract}

\section{Likelihood maximization}

\begin{enumerate}
    % 1.1
    \item
    Suppose we have a data set $ \mathcal{D} \triangleq \{y_1,
    \ldots y_{|\mathcal{D}|}\} $, where each $ y_k \in \mathbb{R} $,
    $ |\mathcal{D}| $ is the number of elements in the set $ \mathcal{D} $.
    Define the loss function $ J_\mathcal{D} : \mathbb{R}
    \rightarrow [0, \infty) $ such that for $ \kappa \in \mathbb{R} $,
    \begin{equation*}
        J_\mathcal{D}(\kappa) = \sum_{k = 1}^{|\mathcal{D}|}(y_k - \kappa)^2
    \end{equation*}
    Show that $ J_\mathcal{D} $ is uniquely minimized by $ \hat{\kappa} $, where
    \begin{equation*}
        \hat{\kappa} = \frac{1}{|\mathcal{D}|}\sum_{k = 1}^{|\mathcal{D}|}y_k
    \end{equation*}
    In other words, show that $ J_\mathcal{D} $ is uniquely minimized by the
    mean of the elements in $ \mathcal{D} $. Use the fact that $ J_\mathcal{D} $
    is strictly convex and thus has a unique minimum.

    % 1.2
    \item
    Suppose we have a data set $ \mathcal{D} \triangleq \{y_1, \ldots
    y_{|\mathcal{D}|}\} $, where each $ y_k \in \{0, 1\} $ is an independent
    realization of the random variable $ Y \sim \mathrm{Bernoulli}(p) $. In
    other words, the underlying distribution for the data is a Bernoulli
    distribution where the probability of a sample having value $ 1 $ is $ p $.
    Since $ Y $ is Bernoulli distributed and since each $ y_k $ is independently
    sampled, the joint likelihood for the data is
    \begin{equation*}
        \ell_\mathcal{D}(\theta) = \prod_{k = 1}^{|\mathcal{D}|}
        \theta^{y_k}(1 - \theta)^{1 - y_k}
    \end{equation*}
    Defining $ N_+ $ as the number of points in $ \mathcal{D} $ with label $ 1 $
    and $ N_- $ as the number of points in $ \mathcal{D} $ with label $ 0 $,
    where necessarily $ N_+ + N_- = |\mathcal{D}| $, show that the maximum
    likelihood estimate of $ p $, $ \hat{\theta} $, is such that
    \begin{equation*}
        \hat{\theta} = \frac{N_+}{N_+ + N_-}
    \end{equation*}
    \textit{Hint.} $ \ell_\mathcal{D} $ has a unique
    maximum. Use the fact that $ \ell_\mathcal{D}(\theta) =
    \theta^{N_+}(1 - \theta)^{N_-} $.

    % 1.3
    \item
    Suppose that the data $ \mathcal{D} $ used in the first problem are
    independent realizations of a scalar random variable $ Y $, where
    $ Y \sim \mathcal{N}(\mu, \sigma^2) $. That is, $ Y $ is normally
    distributed with mean $ \mu $ and variance $ \sigma^2 $. Therefore, the
    joint likelihood of the data, $ \ell_\mathcal{D} $, is such that for some
    $ \theta \in \mathbb{R} $,
    \begin{equation*}
        \ell_\mathcal{D}(\theta) = \prod_{k = 1}^{|\mathcal{D}|}
        \frac{1}{\sqrt{2\pi\sigma^2}}e^{-\frac{1}{2\sigma^2}(y_k - \theta)^2}
    \end{equation*}
    Show that $ \hat{\theta} $, the maximum likelihood estimate for $ \mu $, is
    equal to $ \hat{\kappa} $ from the first problem. In other words, show that
    $ \hat{\theta} $ is the sample mean of the points in $ \mathcal{D} $.

    \medskip

    \textit{Hint.} $ \ell_\mathcal{D} $ obtains a unique maximum at
    $ \hat{\theta} $, i.e. $ \hat{\theta} = \arg\max_\theta
    \ell_\mathcal{D}(\theta) $. But since $ \hat{\theta} $ also uniquely
    minimizes $ \log \circ \ell_\mathcal{D} $, then we may write
    that\footnotemark\footnotetext{
        Since both $ \ell_\mathcal{D} $ and $ \log $ are concave, concavity is
        preserved and the composition $ \log\circ\ell_\mathcal{D} $ is
        concave \cite{bnv}.
    } $ \hat{\theta} = \arg\max_\theta\ell_\mathcal{D}(\theta) =
    \arg\max_\theta\log(\ell_\mathcal{D}(\theta)) $. Write an expression for
    $ \log\circ\ell_\mathcal{D} $ and note that its maximization is equivalent
    to the minimization of its negation $ L_\mathcal{D} \triangleq
    -\log\circ\ell_\mathcal{D} $. $ L_\mathcal{D} $ will turn out to be a an
    affine transformation of $ J_\mathcal{D} $ from the first problem.
\end{enumerate}

\section{Linear algebra}

\begin{enumerate}
    % 2.1
    \item
    Suppose we have a real matrix $ \mathbf{X} \in
    \mathbb{R}^{|\mathcal{D}| \times n} $, where each row $ k $ of the matrix
    corresponds to a single transposed data example $ \mathbf{x}_k^\top $.
    Assuming that each row $ \mathbf{x}_k^\top $ is independently sampled from
    the same vector-valued continuous distribution, the sample mean
    $ \bar{\mathbf{x}} \in \mathbb{R}^n $ of the data points is such that
    \begin{equation*}
        \bar{\mathbf{x}} = \frac{1}{|\mathcal{D}|}\mathbf{X}^\top\mathbf{1}
    \end{equation*}
    Here $ \mathbf{1} \in \mathbb{R}^{|\mathcal{D}|} $, the
    $ \mathbb{R}^{|\mathcal{D}|} $ vector of all $ 1 $s. Show that the centered
    matrix $ \tilde{\mathbf{X}} $, where $ \tilde{\mathbf{X}} \triangleq
    \mathbf{X} - \mathbf{1}\bar{\mathbf{x}}^\top $, may be equivalently written
    as the product of the centering matrix and $ \mathbf{X} $, i.e. that
    \begin{equation*}
        \tilde{\mathbf{X}} = \left(\mathbf{I} - \frac{1}{|\mathcal{D}|}
        \mathbf{11}^\top\right)\mathbf{X}
    \end{equation*}
    Here $ \mathbf{I} $ is the $ |\mathcal{D}| \times |\mathcal{D}| $ identity
    matrix.
\end{enumerate}

\section{Convex functions}

\begin{enumerate}
    % 3.1
    \item
    A convex function $ f : \mathbb{R}^n \rightarrow \mathbb{R} $ satisfies
    $ \forall \alpha \in [0, 1] $, $ \forall \mathbf{x}, \mathbf{y} \in
    \mathbb{R}^n $, the inequality
    \begin{equation} \label{convex_function_def}
        f((1 - \alpha)\mathbf{x} + \alpha\mathbf{y}) \le
        (1 - \alpha)f(\mathbf{x}) + \alpha f(\mathbf{y})
    \end{equation}
    Suppose we define $ f : \mathbb{R}^n \rightarrow
    \mathbb{R} $ to have a quadratic form, where given symmetric
    $ \mathbf{H} \in \mathbb{R}^{n \times n} $, $ \mathbf{a} \in \mathbb{R}^n $,
    \begin{equation} \label{quadratic_form}
        f(\mathbf{x}) = \mathbf{x}^\top\mathbf{Hx} + \mathbf{a}^\top\mathbf{x}
    \end{equation}
    Prove that if $ f $ is convex, then $ \mathbf{H} \succeq \mathbf{0} $, i.e.
    that $ \mathbf{H} $ is positive semidefinite.

    \medskip

    \textit{Hint.} Prove by \textit{modus tollens}, i.e. by proving the
    contrapositive is true. In other words, prove that if $ \mathbf{H} $ is not
    positive semidefinite, then $ f $ is not convex. The easiest way to complete
    the proof is to choose $ \mathbf{H} \in \mathbb{R}^{2 \times 2} $,
    $ \mathbf{x} = \mathbf{a} = \mathbf{0} \in \mathbb{R}^2 $, $ \mathbf{y} =
    \mathbf{1} \in \mathbb{R}^2 $, and define the matrix $ \mathbf{H} $ such
    that for any choice of $ \gamma \ge 1 $,
    \begin{equation*}
        \mathbf{H} \triangleq \mathbf{I} -
        \gamma\begin{bmatrix} \ 0 & 1 \ \\ \ 1 & 0 \ \end{bmatrix}
    \end{equation*}
    Then, substitute the values for $ \mathbf{x}, \mathbf{y}, \mathbf{a},
    \mathbf{H} $ into (\ref{quadratic_form}), compute the quantities
    $ f((1 - \alpha)\mathbf{x}) $, $ (1 - \alpha)f(\mathbf{x}) $,
    $ \alpha f(\mathbf{y}) $ and find a contradiction for the inequality in
    (\ref{convex_function_def}) for some $ \alpha \in  (0, 1) $.

    % 3.2
    \item
    Using the same setup as in the previous problem, prove that if
    $ \mathbf{H} \succeq \mathbf{0} $, then $ f $ is convex.

    \medskip

    \textit{Hint.} Using (\ref{convex_function_def}), compute the quantities
    $ f((1 - \alpha)\mathbf{x} + \alpha\mathbf{y}) $,
    $ (1 - \alpha)f(\mathbf{x}) $, $ \alpha f(\mathbf{y}) $ and show that
    \begin{equation*}
        (1 - \alpha)f(\mathbf{x}) + \alpha f(\mathbf{y}) -
        f((1 - \alpha)\mathbf{x} + \alpha\mathbf{y}) \ge 0
    \end{equation*}
    Use the fact that $ \mathbf{H} \succeq \mathbf{0} $ as the left-hand side
    of the inequality simplifies to a scaled quadratic form. In other words,
    the left-hand side can be written as $ \beta\mathbf{u}^\top\mathbf{Hu} $,
    where $ \mathbf{u} \in \mathbb{R}^n $, $ \beta \in [0, 1) $.

\end{enumerate}

\section{Model analysis}

\begin{enumerate}
    \item
    Suppose we have input matrix $ \mathbf{X} \in
    \mathbb{R}^{|\mathcal{D}| \times n} $ and output vector $ \mathbf{y} \in
    \mathbb{R}^{|\mathcal{D}|} $, where the sample mean of the inputs
    $ \bar{\mathbf{x}} \triangleq
    \frac{1}{|\mathcal{D}|}\mathbf{X}^\top\mathbf{1} = \mathbf{0} $ and the
    sample mean of the outputs $ \bar{y} \triangleq \frac{1}{|\mathcal{D}|}
    \mathbf{y}^\top\mathbf{1} = 0 $, i.e. both the input and output are
    centered. Furthermore, suppose that the sample covariance matrix of the
    input data $ \mathbf{C} = \mathrm{diag}\{\nu_1, \ldots \nu_n\} $,
    $ \nu_i > 0 $. That is, $ \mathbf{C} $ is a diagonal matrix where each
    $ i $th diagonal element is equal to $ \nu_i $, the variance of the $ i $th
    input feature. If we wanted to fit a bias-free\footnotemark\footnotetext{
        If the output and input data are centered, in the case of a linear
        model, no bias is needed. Note that $ \mathbb{E}[\bar{\mathbf{x}}] =
        \mathbb{E}[X] $, $ \mathbb{E}[\bar{y}] = \mathbb{E}[Y] $. Denoting the
        true parameters of the linear model as $ \omega \in \mathbb{R}^n $ and
        the true bias as $ b \in \mathbb{R} $, we see that
        \begin{equation*}
            \mathbb{E}[Y] = \mathbb{E}\big[\omega^\top(X - \bar{x}) + b\big] = b
        \end{equation*}
        Therefore, given centered inputs, the expected value of $ Y $ is $ b $.
        Clearly an unbiased estimator for $ b = \mathbb{E}[Y] $ is $ \bar{y} $,
        the sample mean of the outputs. So if we also center the outputs, we
        see that $ \mathbb{E}[Y - \bar{y}] =
        \mathbb{E}\big[\omega^\top(X - \bar{x})\big] = 0 $, i.e. that no bias
        is needed.
    } linear model to the data, the ordinary least squares (OLS) estimator
    $ \hat{\mathbf{w}} \in \mathbb{R}^n $ of the true weight vector $ \omega $
    is given by
    \begin{equation*}
        \hat{\mathbf{w}} =
        (|\mathcal{D}|\mathbf{C})^{-1}\mathbf{X}^\top\mathbf{y} =
        \frac{1}{|\mathcal{D}|}\mathrm{diag}\{\nu_1^{-1}, \ldots \nu_n^{-1}\}
        \mathbf{X}^\top\mathbf{y}
    \end{equation*}
    For some $ \lambda \ge 0 $, the ridge regression estimator
    $ \hat{\mathbf{w}}_\lambda \in \mathbb{R}^n $ of the true weight vector is
    given by
    \begin{equation*}
        \begin{split}
            \hat{\mathbf{w}}_\lambda =
            (|\mathcal{D}|\mathbf{C} + \lambda\mathbf{I})^{-1}
            \mathbf{X}^\top\mathbf{y} & =
            \mathrm{diag}\left\{
                \frac{1}{|\mathcal{D}|\nu_1 + \lambda}, \ldots
                \frac{1}{|\mathcal{D}|\nu_n + \lambda}
            \right\}\mathbf{X}^\top\mathbf{y} \\
            & = \mathrm{diag}\left\{
                \frac{|\mathcal{D}|\nu_1}{|\mathcal{D}|\nu_1 + \lambda},
                \ldots \frac{|\mathcal{D}|\nu_n}{|\mathcal{D}|\nu_n + \lambda}
            \right\}\hat{\mathbf{w}}
        \end{split}
    \end{equation*}
    What is $ \lim_{\lambda \rightarrow \infty}\hat{\mathbf{w}}_\lambda $? For
    a fixed value of $ \lambda $, explain how for each $ i $th input component,
    the variance $ \nu_i $ affects how the corresponding OLS coefficient $ w_i $
    is deflated by ridge regression.

    \item
    The AdaBoost\footnotemark\footnotetext{
        Short for \textit{adaptive boosting}.
    } algorithm is an example of a \textit{boosting algorithm}, which is a
    supervised learning algorithm that fits an additive model using a greedy
    stagewise approach. The original \textit{discrete AdaBoost} formulation fits
    a classifier $ G : \mathbb{R}^n \rightarrow \{-1, 1\} $,
    $ G(\mathbf{x}) = \mathrm{sgn}(F(\mathbf{x})) $, where
    $ F : \mathbb{R}^n \rightarrow \mathbb{R} $ is of the form
    \begin{equation*}
        F(\mathbf{x}) \triangleq F_M(\mathbf{x}) =
        \sum_{m = 1}^M\alpha_mf_m(\mathbf{x})
    \end{equation*}
    Here $ \alpha_1, \ldots \alpha_M \in \mathbb{R} $ and
    $ f_1, \ldots f_M \in \mathcal{H} $, $ \mathcal{H} $ some function subspace.
    Each $ f_m $ is a classifier outputting a label in $ \{-1, 1\} $. At each
    boosting iteration $ m $, given the partially built additive model
    $ F_{m - 1} $, discrete AdaBoost attempts to choose the optimal
    $ \alpha_m, f_m $ such that $ F_m \triangleq F_{m - 1} + \alpha_mf_m $. It
    does so by minimizing the $ m $-th iteration loss functional
    $ J_{m, \alpha} $, where
    \begin{equation*}
        J_{m, \alpha}[f] = \sum_{k = 1}^{|\mathcal{D}|}
        L(y_k, F_{m - 1}(\mathbf{x}_k) + \alpha f(\mathbf{x}_k))
    \end{equation*}
    Here $ L : \mathbb{R}^2 \rightarrow [0, \infty) $ is the exponential loss
    function, where $ L(y, \hat{y}) = e^{-y\hat{y}} $. Expanding, one sees that
    $ J_{m ,\alpha} $ can be rewritten as a sum of weighted exponential
    losses $ w_{m, k}e^{-y_k\alpha f(\mathbf{x}_k)} $,
    $ k \in \{1, \ldots |\mathcal{D}|\} $, where $ w_{m, k} \triangleq
    e^{-y_kF_{m - 1}(\mathbf{x}_k)} $. Therefore, we can write $ J_{m, \alpha} $
    as
    \begin{equation} \label{adaboost_functional}
        J_{m, \alpha}[f] = \sum_{k = 1}^{|\mathcal{D}|}w_{m, k}
        e^{-y_k\alpha f(\mathbf{x}_k)} =
        \sum_{k = 1}^{|\mathcal{D}|}w_{m, k}(e^\alpha)^{-y_kf(\mathbf{x}_k)}
    \end{equation}
    $ J_{m, \alpha} $ is convex, and since $ \alpha $ only changes the
    exponentiation base, then the optimal $ f_m $ is simply the solution to
    $ \min_{f \in \mathcal{H}}J_{m, 1}[f] $, minimizing a sum of weighted
    exponential losses. With $ f_m $ known, $ \alpha_m $ can be found in closed
    form. Defining
    $ I_{f_m} \triangleq \{k \in \{1, \ldots |\mathcal{D}|\} :
    y_k = f_m(\mathbf{x}_k)\} $, the set of indices where
    $ y_k = f_m(\mathbf{x}_k) $, $ k \in \{1, \ldots |\mathcal{D}|\} $, then
    $ \alpha_m $ is such that
    \begin{equation} \label{adaboost_alpha}
        \alpha_m = \frac{1}{2}\log\left(
            \frac{\sum_{k \in I_{f_m}}w_{m, k}}{
                \sum_{k \in I_{f_m}^\mathsf{c}}w_{m, k}
            }
        \right)
    \end{equation}
    Derive (\ref{adaboost_alpha}) by starting from (\ref{adaboost_functional}).

    \medskip

    \textit{Hint.} Use the fact that $ J_{m, \alpha}[f_m] $ is convex in
    $ \alpha $. Note that $ I_{f_m}^\mathsf{c} \triangleq
    \{k \in \{1, \ldots |\mathcal{D}|\} : y_k \ne f_m(\mathbf{x}_k)\} $, and so
    $ I_{f_m} \cup I_{f_m}^\mathsf{c} = \{1, \ldots |\mathcal{D}|\} $. It is
    also important to note that $ y_k, f_m(\mathbf{x}_k) \in \{-1, 1\} $,
    $ \forall k \in \{1, \ldots |\mathcal{D}|\} $. Therefore,
    $ y_kf_m(\mathbf{x}_k) = 1 $ if and only if $ y_k = f_m(\mathbf{x}_k) $ and
    is $ -1 $ otherwise. The resulting expression must only involve the $ m $th
    iteration weights $ w_{m, k} $.

\end{enumerate}

\begin{thebibliography}{99}
    \bibitem{bnv}
    Boyd, S., \& Vandenberghe, L. (2004). \textit{Convex Optimization}.
    Cambridge University Press.
\end{thebibliography}

\end{document}