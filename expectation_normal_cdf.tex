\documentclass{article}
\usepackage[margin=1in]{geometry}
\usepackage{amsmath, amssymb, amsfonts}
% change equation numbering to section.eq_num
\numberwithin{equation}{section}
% set paragraph indent
\setlength\parindent{0pt}
% makes clickable links to sections
\usepackage{hyperref}
% make the link colors blue, as well as cite colors
\hypersetup{
    colorlinks, linkcolor = blue, citecolor = blue, urlcolor = magenta
}

\title{A normal cdf expectation}
\author{Derek Huang}
\date{January 2, 2021\thanks{Original published June 2019.}}

\begin{document}

\maketitle

\section{Introduction}

A couple years ago, I had a problem that involved the computation of the
expectation of a function of a normal random variable, where the function in
question was the cumulative distribution function of a different, independent
normal random variable. At the time, I wrote a 3-page article that mostly
consisted of me rewriting normal random variables as affine transformations of
other normal random variables, which consisted of much rote derivation and
frankly was not too interesting.

\medskip

This article is a much shorter version of the original. The problem has also
been trivially modified to be more general by dropping the assumption that the
normal random variables $ X, Y $ are independent.

\section{A solution}

Suppose we have $ X \sim \mathcal{N}(\mu, \sigma^2) $,
$ Y \sim \mathcal{N}(\nu, \tau^2) $, with correlation
$ \rho_{X, Y} \in [-1, 1] $. We want to compute the quantity
$ \mathbb{E}[F_X(Y)] $, where $ F_X $ is the cdf of $ X $, i.e. for
$ x \in \mathbb{R} $,
\begin{equation*}
    F_X(x) = \Phi\left(\frac{x - \mu}{\sigma}\right)
\end{equation*}

Here $ \Phi $ is the standard normal cdf. By the definition of expectation for a
continous random variable and by the law of total probability, we can
conveniently rewrite $ \mathbb{E}[F_X(Y)] $ as
\begin{equation*}
    \mathbb{E}[F_X(Y)] = \int_\mathbb{R}F_X(y)\phi_Y(y)\,dy =
    \int_\mathbb{R}\mathbb{P}\{X \le Y \mid Y = y\}\phi_Y(y)\,dy =
    \mathbb{P}\{X \le Y\}
\end{equation*}

Since $ \mathbb{P}\{X \le Y\} = \mathbb{P}\{X - Y \le 0\} $, we just need the
distribution of $ X - Y $ and we can compute $ \mathbb{E}[F_X(Y)] $. Since
$ X, Y $ are both normal random variables, we know
$ \mathbb{E}[X - Y] = \gamma \triangleq \mu - \nu $. However, since $ X, Y $ are
not necessarily independent, using properties of
variance\footnotemark\footnotetext{
    We can verify the first equality manually at the cost of some uninteresting
    work. Using the definition of variance,
    \begin{equation*}
        \begin{split}
            \mathrm{Var}(X - Y) & = \mathbb{E}\big[(X - Y - \gamma)^2\big] =
            \mathbb{E}\big[(X - Y)^2 - 2\gamma(X - Y) + \gamma^2\big] =
            \mathbb{E}\big[(X - Y)^2\big] - \gamma^2 \\
            & = \mathbb{E}\big[X^2 - 2XY + Y^2\big] - \mu^2 + 2\mu\nu - \nu^2 =
            \mathbb{E}\big[X^2] - \mu^2 + \mathbb{E}\big[Y^2\big] - \nu^2 -
            2(\mathbb{E}[XY] - \mu\nu) \\
            & = \mathrm{Var}(X) + \mathrm{Var}(Y) - 2\mathrm{cov}(X, Y)
        \end{split}
    \end{equation*}
},
\begin{equation*}
    \mathrm{Var}(X - Y) = \mathrm{Var}(X) + \mathrm{Var}(Y) -
        2\mathrm{cov}(X, Y) =
    \sigma^2 + \tau^2 - 2\rho_{X, Y}\sigma\tau
\end{equation*}

We can then rewrite $ F_{X - Y} $, the cdf of $ X - Y $, using the standard
normal cdf. Therefore,
\begin{equation} \label{solution}
    \mathbb{E}[F_X(Y)] = F_{X - Y}(0) =
    \Phi\left(-\frac{\mu - \nu}{
        \sqrt{\sigma^2 + \tau^2 - 2\rho_{X, Y}\sigma\tau}
    }\right)
\end{equation}

In the typical scenario, $ Y $ is usually independent of $ X $, so the
correlation term in (\ref{solution}) may be omitted.

\end{document}