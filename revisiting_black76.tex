\documentclass{article}
\usepackage[margin = 1in]{geometry}
\usepackage{amsmath, amssymb, graphicx}
% change equation numbering to section.eq_num + set paragraph indent to 0
\numberwithin{equation}{section}
\setlength\parindent{0pt}
% makes clickable links to sections
\usepackage{hyperref}
% make the link colors blue, as well as cite colors
\hypersetup{
    colorlinks, linkcolor = blue, citecolor = blue, urlcolor = magenta
}
% set up figure caption format (can get rid of this later)
\usepackage{caption}
\captionsetup{font = small}

\title{Revisiting Black's model}
\author{Derek Huang}
\date{January 11, 2021\thanks{
    Previous version dated August 7, 2019. Notation has been cleaned up and the
    post-derivation analysis has been reworked.
}}

\begin{document}

\maketitle

\begin{abstract}
    Solving the Black-Scholes PDE by means of transformation to the heat
    equation\footnotemark\footnotetext{
        Derivation of the Black-Scholes formula using this method can be found
        in Wilmott, Howison, and Dewynne's \textit{The Mathematics of Financial
        Derivatives: A Student Introduction}. This book was the author's first
        exposure to quantitative finance.
    }
    proffers little financial intuition on the form of the pricing formula. An
    alternate approach using the risk-neutral pricing method allows for a
    simpler, more direct derivation of the formula, and allows one to re-express
    the formula in terms of risk-neutral conditional probabilities that shed a
    probabilistic light on the replication argument. We then examine the
    dynamics of the at-the-money European call under Black's model with regards
    to strike and maturity using both simple mathematics and computer-generated
    graphs.
\end{abstract}

\section{Derivation}

\subsection{Risk-neutral pricing}

The futures price process $ \{F_t\}_{t \in \mathbb{R}_+} $ under the
risk-neutral measure $ \mathbb{Q} $ in Black's model satisfies
\begin{equation} \label{black_dgbm}
    dF_t = \sigma F_t\,dW_t
\end{equation}

Here $ W_t $ is a Wiener process under $ \mathbb{Q} $, and we know that
$ W_t \sim \mathcal{N}(0, t) $. By the risk-neutral pricing formula, we can
write the time $ t $ value $ C_{t, T}^F $ of a European call option on $ F $ as
the conditional expectation
\begin{equation*}
    C_{t, T}^F = \mathbb{E}[D_{t, T}\max\{F_T - K, 0\} \mid \mathcal{F}_t]
\end{equation*}

Here $ F_t $ is the current futures price, $ T $ is the time of option
expiration, and $ K $ is the fixed option strike. $ D_{t, T} $ is a discount
factor, which Black specifies as the deterministic $ e^{-r(T - t)} $, with
$ r $ being a known, constant riskless discount rate. $ \mathcal{F}_t $ is the
time-$ t $ sub $ \sigma $-algebra from the natural filtration generated by the
process $ \{F_t\}_{t \in \mathbb{R}_+} $. Since Black's $ D_{t, T} $ is
deterministic, we can write the risk-neutral conditional expectation as
\begin{equation*}
    C_{t, T}^F = D_{t, T}\int_\mathbb{R}\max\{f - K, 0\}
    \varphi_{F_T, \tau \mid \mathcal{F}_t}(f)\,df =
    D_{t, T}\int_K^\infty(f - K)\varphi_{F_T, \tau \mid \mathcal{F}_t}(f)\,df
\end{equation*}

Here $ \varphi_{F_T, \tau \mid \mathcal{F}_t} $ is the risk-neutral probability
density for $ F_T $ conditional on $ \mathcal{F}_t $, where $ \tau \ge T \ge t $
is a delivery date. For brevity, we drop the explicit dependence on
$ \mathcal{F}_t $, $ T $ and use the simpler notation
$ \widetilde{\varphi}_\tau $ instead. Dividing by $ D_{t, T} $ and expanding the
remaining integral, we have
\begin{equation} \label{black_udE}
    \begin{split}
        \frac{C_{t, T}^F}{D_{t, T}} & = \int_K^\infty(f - K)
        \widetilde{\varphi}_\tau(f)\,df =
        \int_K^\infty f\widetilde{\varphi}_\tau(f)\,df - 
        K\int_K^\infty\widetilde{\varphi}_\tau(f)\,df \\ & =
        \int_K^\infty f\widetilde{\varphi}_\tau(f)\,df - 
        K\mathbb{Q}\{F_T \ge K \mid \mathcal{F}_t\}
    \end{split}
\end{equation}

We see that the expectation can be written as the difference between a partial
expectation of $ F_T $ conditional on $ \mathcal{F}_T $ and $ K $ times a
risk-neutral probability of a $ \mathcal{F}_t $-measurable event. To simplify
our computation of the partial expectation, we recall that the solution to
(\ref{black_dgbm}) is given by the equation
\begin{equation*}
    F_t = F_0e^{-\frac{1}{2}\sigma^2t + \sigma W_t}
\end{equation*}

Now we want to find an analytical expression for $ F_T $ conditioned on
$ \mathcal{F}_t $. By the properties of the Wiener process, we write
$ W_T \mid \mathcal{F}_t $ as $ \sqrt{T - t}Z $, where
$ Z \sim \mathcal{N}(0, 1) $ under $ \mathbb{Q} $. We then have
\begin{equation} \label{black_FT_tad}
    F_T \mid \mathcal{F}_t = F_te^{-\frac{1}{2}\sigma^2(T - t) + 
    \sigma\sqrt{T - t}Z}
\end{equation}

For notational simplicity, set $ \theta \triangleq T - t $. To simplify the
partial expectation, we make note that $ F_T \mid \mathcal{F}_t $ is a function
of a standard normal random variable $ Z $. Therefore, using
(\ref{black_FT_tad}), we can rewrite the partial expectation in terms of the
standard normal density function $ \phi $ instead to get
\begin{equation} \label{black_pe_1}
    \int_K^\infty f\widetilde{\varphi}_\tau(f)\,df =
    F_t\int_\zeta^\infty e^{-\frac{1}{2}\sigma^2\theta +
    \sigma\sqrt{\theta}z}\phi(z)\,dz =
    F_t\int_\zeta^\infty\frac{1}{\sqrt{2\pi}}e^{-\frac{1}{2}\sigma^2\theta +
    \sigma\sqrt{\theta}z - \frac{1}{2}z^2}\,dz
\end{equation}

The last step follows from the definition of the standard normal density
$ \phi $, where of course
\begin{equation*}
    \phi(z) \triangleq \frac{1}{\sqrt{2\pi}}e^{-\frac{1}{2}z^2}
\end{equation*}

Since the partial expectation is taken over $ Z $ instead of
$ F_T \mid \mathcal{F}_t $, a function of $ Z $, the threshold $ K $ changes to
a different quantity $ \zeta $. We find $ \zeta $ by solving the equation
\begin{equation*}
    K = F_te^{-\frac{1}{2}\sigma^2\theta + \sigma\sqrt{\theta}\zeta}
\end{equation*}

This relates $ \zeta $ to $ K $ through the relation between
$ F_T \mid \mathcal{F}_t $ and $ Z $ given by (\ref{black_FT_tad}), so we see
that
\begin{equation} \label{black_zeta}
    \zeta = \frac{
        \log(K / F_t) + \frac{1}{2}\sigma^2\theta
    }{\sigma\sqrt{\theta}}
\end{equation}

Here $ \log $ is the natural logarithm. Since the exponentiated quadratic in
(\ref{black_pe_1}) is a perfect square, then
\begin{equation*}
    \int_K^\infty f\widetilde{\varphi}_\tau(f)\,df =
    F_t\int_\zeta^\infty\frac{1}{\sqrt{2\pi}}
    e^{-\frac{1}{2}(z - \sigma\sqrt{\theta})^2}dz = 
    F_t\mathbb{Q}\{X \ge \zeta\}
\end{equation*}

Here $ X \sim \mathcal{N}(\sigma\sqrt{\theta}, 1) $, as we see that the function
being integrated above is a normal density. Since we can write
$ X = Z + \sigma\sqrt{\theta} $, if we rewrite $ \mathbb{Q}\{X \ge \zeta\} $
in terms of $ Z $ and substitute for $ \zeta $ the result in (\ref{black_zeta}),
we have
\begin{equation} \label{black_p1}
    \begin{split}
        \mathbb{Q}\{X \ge \zeta\} & =
        \mathbb{Q}\left\{Z + \sigma\sqrt{\theta} \ge \frac{\log(K / F_t) +
        \frac{1}{2}\sigma^2\theta}{\sigma\sqrt{\theta}}\right\} =
        \mathbb{Q}\left\{Z \ge \frac{
            \log(K / F_t) - \frac{1}{2}\sigma^2\theta
        }{\sigma\sqrt{\theta}}\right\} \\ & =
        \mathbb{Q}\left\{Z \le \frac{
            \log(F_t / K) + \frac{1}{2}\sigma^2\theta
        }{\sigma\sqrt{\theta}}\right\} =
        \Phi\left(\frac{
            \log(F_t / K) + \frac{1}{2}\sigma^2\theta
        }{\sigma\sqrt{\theta}}\right)
    \end{split}
\end{equation}

Here $ \Phi $ is the standard normal cdf and the second to last equality follows
from the symmetry of $ Z $ around $ 0 $. Using our result from (\ref{black_p1})
and replacing $ \theta $ with $ T - t $, we have
\begin{equation} \label{black_partial_E}
    \int_K^\infty f\widetilde{\varphi}_\tau(f)\,df =
    F_t\Phi\left(\frac{
        \log(F_t / K) + \frac{1}{2}\sigma^2(T - t)
    }{\sigma\sqrt{T - t}}\right)
\end{equation}

This is the first normal cdf term in Black's formula. Moving on to the
risk-neutral conditional probability
$ \mathbb{Q}\{F_T \ge K \mid \mathcal{F}_t\} $ we defined in (\ref{black_udE}),
we again use our expression for $ F_T \mid \mathcal{F}_t $ in
(\ref{black_FT_tad}). Substituting $ \theta $ for $ T - t $ as before, we can
rewrite $ \mathbb{Q}\{F_T \ge K \mid \mathcal{F}_t\} $ in terms of $ Z $ and
find that
\begin{equation} \label{black_p2}
    \begin{split}
        \mathbb{Q}\{F_T \ge K \mid \mathcal{F}_t\} & =
        \mathbb{Q}\left\{F_te^{
            -\frac{1}{2}\sigma^2\theta + \sigma\sqrt{\theta}Z
        } \ge K\right\} =
        \mathbb{Q}\left\{Z \ge \frac{
            \log(K / F_t) + \frac{1}{2}\sigma^2\theta
        }{\sigma\sqrt{\theta}}\right\} \\ & =
        \mathbb{Q}\left\{Z \le \frac{
            \log(F_t / K) - \frac{1}{2}\sigma^2\theta
        }{\sigma\sqrt{\theta}}\right\} =
        \Phi\left(\frac{
            \log(F_t / K) - \frac{1}{2}\sigma^2(T - t)
        }{\sigma\sqrt{T - t}}\right)
    \end{split}
\end{equation}

Our derivation has now been completed. Substituting our results from
(\ref{black_partial_E}) and (\ref{black_p2}) back into (\ref{black_udE}) and
multiplying by $ D_{t, T} = e^{-r(T - t)}$, we get exactly Black's formula
\begin{equation} \label{black_call}
    C_{t, T}^F = D_{t, T}\left[F_t\Phi\left(\frac{\log(F_t / K) +
    \frac{1}{2}\sigma^2(T - t)}{\sigma\sqrt{T - t}}\right) -
    K\Phi\left(\frac{
        \log(F_t / K) - \frac{1}{2}\sigma^2(T - t)
    }{\sigma\sqrt{T - t}}\right)\right]
\end{equation}

\subsection{$ \mathcal{F}_t $-conditional replication}

By deriving Black's formula from the risk-neutral pricing formula, we can
re-express the normal cdf terms as risk-neutral conditional probabilities in
terms of $ F_T $, which is the inverse of what was done when deriving the
formula. Since (\ref{black_p2}) already shows how the second normal cdf term is
written as a $ \mathcal{F}_t$-conditional $ \mathbb{Q} $ probability involving
$ F_T $, we work on the first normal cdf term. From (\ref{black_FT_tad}), we
have
\begin{equation} \label{Z_FT}
    Z = \left.\frac{
        \log(F_T / F_t) + \frac{1}{2}\sigma^2(T - t)
    }{\sigma\sqrt{T - t}} \ \middle\vert \ \mathcal{F}_t\right.
\end{equation}

Writing $ \theta = T - t $ as before, we can re-express the first normal cdf
term in (\ref{black_call}) using (\ref{Z_FT}) as
\begin{equation} \label{black_q1}
    \begin{split}
        \Phi\left(\frac{
            \log(F_t / K) + \frac{1}{2}\sigma^2(T - t)
        }{\sigma\sqrt{T - t}}\right) & =
        \mathbb{Q}\left\{Z \le \frac{
            \log(F_t / K) + \frac{1}{2}\sigma^2\theta
        }{\sigma\sqrt{\theta}}\right\} =
        \mathbb{Q}\left\{Z \ge \frac{
            \log(K / F_t) - \frac{1}{2}\sigma^2\theta
        }{\sigma\sqrt{\theta}}\right\} \\ & =
        \mathbb{Q}\left\{\frac{
            \log(F_T / F_t) + \frac{1}{2}\sigma^2\theta
        }{\sigma\sqrt{\theta}} \ge \frac{
            \log(K / F_t) - \frac{1}{2}\sigma^2\theta
        }{\sigma\sqrt{\theta}} \ \middle\vert \ \mathcal{F}_t\right\} \\ & =
        \mathbb{Q}\left\{F_T \ge Ke^{-\sigma^2(T - t)} \ \middle\vert \ 
        \mathcal{F}_t\right\}
    \end{split}
\end{equation}

Substituting our results from (\ref{black_p2}) and (\ref{black_q1}) into Black's
formula given in (\ref{black_call}), we have
\begin{equation} \label{black_qcall}
    C_{t, T}^F = D_{t, T}\left[F_t\mathbb{Q}\left\{F_T \ge Ke^{-\sigma^2(T - t)}
    \ \middle\vert \ \mathcal{F}_t\right\} -
    K\mathbb{Q}\{F_T \ge K \mid \mathcal{F}_t\}\right]
\end{equation}

A European call option can be replicated using the underlying and a risk-free
asset, in this case $ D_{t, T} $, which is essentially a
risk-free zero-coupon bond. Expressing Black's formula with (\ref{black_qcall})
helps show how the quantity of the underlying to long and quantity of the
risk-free asset to short are dependent on $ \mathcal{F}_t $-conditional
$ \mathbb{Q} $ probabilities of $ F_T $ exceeding a quantity involving the
strike $ K $. The first $ \mathbb{Q} $ probability can be seen from inspection
to be greater than or equal to the second, strictly so when
$ \sigma^2(T - t) > 0 $. Both $ \mathbb{Q} $ probabilities can also be
interpreted as $ \mathcal{F}_t $-conditional risk-neutral exercise
probabilities.

\section{Properties}

\subsection{Limiting behavior}

From (\ref{black_call}) we can easily verify well-known limiting behavior of
the European call. For example\footnotemark\footnotetext{
    We used the \textit{undiscounted} value $ C_{t, T}^F / D_{t, T} $ when
    taking the $ T \rightarrow \infty $ limit, as
    $ \lim_{T \rightarrow \infty}D_{t, T} = 0 $. In comparison, under the
    Black-Scholes model, for an underlying $ S $ that pays no dividends, the
    value $ C_{t, T}^S $ of the European call as $ T \rightarrow \infty $
    converges to $ S_t $, the current price of the underlying. Other limits
    that involved the undiscounted value $ C_{t, T}^F / D_{t, T} $ serve to
    remove the discount factor from the right-hand side of the limit equality.
},
\begin{equation} \label{black_limits}
    \begin{split}
        \lim_{t \rightarrow T_-}C_{t, T}^F & =
        \mathbb{I}\{F_t \ge K\}(F_t - K) \\
        \lim_{T \rightarrow \infty}\frac{C_{t, T}^F}{D_{t, T}} & = F_t \\
        \lim_{\sigma \rightarrow \infty}\frac{C_{t, T}^F}{D_{t, T}} & = F_t \\
        \lim_{F_t / K \rightarrow \infty}\frac{C_{t, T}^F}{D_{t, T}} & =
        (F_t - K) \\
        \lim_{F_t / K \rightarrow 0_+}C_{t, T}^F & = 0
    \end{split}
\end{equation}

Here $ \mathbb{I} $ is the indicator function. None of the limits shown in
(\ref{black_limits}) should be too surprising as they express well-known facts
about how European options should behave in general, although the fact that the
undiscounted European call value $ C_{t, T}^F / D_{t, T} $ converges to $ F_t $
as $ \sigma \rightarrow \infty $ is interesting but not very useful.

\subsection{At-the-money dynamics}

We can also see that under the Black model, the at-the-money European call,
whose value we will denote by $ \hat{C}_{t, T}^F $, has relatively simple
dynamics in terms of strike and maturity. From (\ref{black_call}), we can write
\begin{equation} \label{black_atm_call}
    \hat{C}_{t, T}^F = D_{t, T}K\xi(\sigma, T - t)
\end{equation}

As expected, the function $ \xi : [0, \infty)^2 \rightarrow [0, 1] $ is defined
such that
\begin{equation} \label{black_xi}
    \xi(\sigma, \theta) = \Phi\left(\frac{1}{2}\sigma\sqrt{\theta}\right) -
    \Phi\left(-\frac{1}{2}\sigma\sqrt{\theta}\right)
\end{equation}

From inspection, we see that under the Black model, the price of the
at-the-money European call is linear in the strike $ K $. Also, we see that
$ \forall K' \in [0, \infty) $,
$ \partial \hat{C}_{t, T}^F / \partial K \vert_{K = K'} \in [0, D_{t, T}] $, as
$ \forall (u, v) \in [0, \infty)^2, \xi(u, v) \in [0, 1] $. The dynamics of the
at-the-money European call with respect to maturity $ T $ are not immediately
obvious from inspection, however, so we compute
$ \partial \hat{C}_{t, T}^F / \partial T $, where from (\ref{black_atm_call}),
(\ref{black_xi}) we see that
\begin{equation*}
    \begin{split}
        \frac{\partial \hat{C}_{t, T}^F}{\partial T} & =
        -rD_{t, T}K\xi(\sigma, T - t) + D_{t, T}K\left[
            \phi\left(
                \frac{1}{2}\sigma\sqrt{T - t}
            \right)\frac{\sigma}{4\sqrt{T - t}} +
            \phi\left(
                -\frac{1}{2}\sigma\sqrt{T - t}
            \right)\frac{\sigma}{4\sqrt{T - t}}
        \right] \\ & =
        D_{t, T}K\left[
            -r\xi(\sigma, T - t) + \frac{1}{2}\sigma(T - t)^{-1 / 2}\phi\left(
                \frac{1}{2}\sigma\sqrt{T - t}
            \right)
        \right]
    \end{split}
\end{equation*}

Here we used the fact that $ D_{t, T} \triangleq e^{-r(T - t)} $, so
$ \partial D_{t, T} / \partial T = -re^{-r(T - t)} = -rD_{t, T} $, and the fact
that the standard normal density $ \phi $ is symmetric around the origin, i.e.
$ \forall z \in \mathbb{R} $, $ \phi(z) = \phi(-z) $. We see that
\begin{equation*}
    \lim_{T \rightarrow \infty} \partial\hat{C}_{t, T}^F / \partial T =
    -rK\lim_{T \rightarrow \infty}D_{t, T}\xi(\sigma, T - t) +
    \frac{1}{2}\sigma K\lim_{T \rightarrow \infty}D_{t, T}(T - t)^{-1 / 2}
    \phi\left(\frac{1}{2}\sigma\sqrt{T - t}\right) = 0
\end{equation*}
also that
\begin{equation*}
    \lim_{T - t \rightarrow 0_+} \partial\hat{C}_{t, T}^F / \partial T =
    -rK\lim_{T - t \rightarrow 0_+}D_{t, T}\xi(\sigma, T - t) +
    \frac{1}{2}\sigma K\lim_{T - t \rightarrow 0_+}D_{t, T}(T - t)^{-1 / 2}
    \phi\left(\frac{1}{2}\sigma\sqrt{T - t}\right)
\end{equation*}
To compute the above limits, we used the fact that the limit of the product of
continuous functions exists when the individual limits of the continous
functions exist and is equal to the product of the individual limits. The
individual limits used are shown below.
\begin{equation*}
    \begin{split}
        \lim_{T \rightarrow \infty}D_{t, T} & = 0 \\
        \lim_{T - t \rightarrow 0_+}D_{t, T} & = 1 \\
        \lim_{T \rightarrow \infty}\xi(\sigma, T - t) & =
        \mathbb{I}\{\sigma > 0\} \\
        \lim_{T - t \rightarrow 0_+}\xi(\sigma, T - t) & = 0 \\
    \end{split}
\end{equation*}
$
 = \infty
$\footnotemark\footnotetext{
    Since $ \lim_{T - t \rightarrow 0_+}D_{t, T} = 1 $ and
    $ \forall \sigma \in [0, \infty) $,
    $ \lim_{T - t \rightarrow 0_+}\xi(\sigma, T - t) = 0 $, then
    $ -\lim_{T - t \rightarrow 0_+}rKD_{t, T}\xi(\sigma, T - t) = 0 $.
}.

% The intuition here is that under the Black model, the value of a European call
% option is the discounted difference between the current value of the underlying,
% here the futures price $ F_t $, and the strike $ K $, each weighted by a
% risk-neutral probability. Interestingly, we may also write (\ref{black_call}) as
% as\footnotemark\footnotetext{
%     We use the obvious fact that
%     \begin{equation*}
%         \mathbb{Q}\left\{F_T \ge Ke^{-\sigma^2(T - t)} \ \middle\vert \
%         \mathcal{F}_t\right\} =
%         \mathbb{Q}\{F_T \ge K \mid \mathcal{F}_t\} +
%         \mathbb{Q}\left\{F_T \in \big[Ke^{-\sigma^2(T - t)}, K\big] \
%         \middle\vert \ \mathcal{F}_t\right\}
%     \end{equation*}
% }
% \begin{equation} \label{black_qcall_p}
%     C_{t, T}^F = D_{t, T}\left[
%         (F_t - K)\mathbb{Q}\{F_T \ge K \mid \mathcal{F}_t\} +
%         F_t\mathbb{Q}\left\{
%             F_T \in \big[Ke^{-\sigma^2(T - t)}, K\big] \ \middle\vert \
%             \mathcal{F}_t
%         \right\}
%     \right]
% \end{equation}
% (\ref{black_qcall_p}) shows that under the Black model, the value of a European
% call is equal to the the sum of two quantities. The first is the risk-neutral
% present value of a long forward contract\footnotemark\footnotetext{
%     The present value of a long forward contract on $ F $ with
%     forward (delivery) price $ K $ and delivery date $ T $ under $ \mathbb{Q} $
%     simply is, by the risk-neutral pricing formula,
%     $ \mathbb{E}[D_{t, T}(F_T - K)] = D_{t, T}(F_t - K) $. This is because
%     $ \{F_t\}_{t \in \mathbb{R}_+} $ is a martingale.
% } on the underlying futures with delivery
% date $ T $ and forward price $ K $ times the option's conditional exercise
% probability $ \mathbb{Q}\{F_T \ge K \mid \mathcal{F}_t\}$. The second is the
% discounted expectation of the terminal price $ F_T $ times the
% $ \mathcal{F}_t $-conditional probability that $ F_T $ falls within the interval
% $ \big[Ke^{-\sigma^2(T - t)}, K\big] $. For an underlying with higher
% volatility, this interval is wider, although it is clear that as the option
% approaches maturity, the interval converges to $ K $. We can rewrite this


% The first exercise probability
% $ \mathbb{Q}\big\{F_T \ge Ke^{-\sigma^2(T - t)}\big\} $ has the strike weighted
% by the quantity $ e^{-\sigma^2(T - t)} $ such that \textit{ceteris paribus}, if
% $ \sigma $ increases, $ C_{t, T}^F $ increases. This corroborates the intuition
% that \textit{ceteris paribus}, if an asset has high volatility, an option on
% that asset will be worth more.

% \textit{ceteris paribus}, as volatility increases, the value of an option
% increases. We can quantify this by separating out the volatility
% premium\footnotemark\footnotetext{
%     Here volatility is used to refer to the \textit{total volatility}, i.e. the
%     instantaneous diffusion coefficient $ \sigma $ multiplied by
%     $ \sqrt{T - t} $, the square root of the time to maturity. The option time
%     premium is implicitly included in this quantity.
% }, as
% \begin{equation*}
%     \begin{split}
%         \mathbb{Q}\left\{F_T \ge Ke^{-\sigma^2(T - t)} \ \middle\vert \
%         \mathcal{F}_t\right\} & =
%         \Phi\left(\frac{
%             \log(F_t / K) + \frac{1}{2}\sigma^2\theta
%         }{\sigma\sqrt{\theta}}\right) =
%         \mathbb{Q}\left\{Z \le - \zeta + \sigma\sqrt{\theta}\right\} =
%         \int_{-\infty}^{-\zeta + \sigma\sqrt{\theta}}\phi(z)\,dz \\
%         \mathbb{Q}\{F_T \ge K \mid \mathcal{F}_t\} & =
%         \Phi\left(\frac{
%             \log(F_t / K) - \frac{1}{2}\sigma^2\theta
%         }{\sigma\sqrt{\theta}}\right) = \mathbb{Q}\{Z \le -\zeta\} =
%         \int_{-\infty}^{-\zeta}\phi(z)\,dz
%     \end{split}
% \end{equation*}

% Here we have $ \zeta $ as defined in (\ref{black_zeta}), $ \theta = T - t $ as
% before, and $ \phi $ the normal pdf. We can thus see that
% \begin{equation*}
%     \mathbb{Q}\left\{F_T \ge Ke^{-\sigma^2(T - t)} \ \middle\vert \
%     \mathcal{F}_t\right\} =
%     \mathbb{Q}\{F_T \ge K \mid \mathcal{F}_t\} +
%     \int_{-\zeta}^{-\zeta + \sigma\sqrt{\theta}}\phi(z)\,dz
% \end{equation*}

% Therefore, we can write

% We can therefore define a function of the current futures price $ F_t $ and
% current time $ t $ parametrized by $ \sigma $,
% $ \lambda_\sigma : [0, \infty)^2 \rightarrow [0, \infty) $, where
% \begin{equation} \label{black_vol_prem}
%     \lambda_\sigma(F_t, \theta) =
%     \int_{-\zeta}^{-\zeta + \sigma\sqrt{\theta}}\phi(z)\,dz, \quad
%     \text{where} -\zeta =
%     \frac{\log(F_t / K) - \frac{1}{2}\sigma^2\theta}{\sigma\sqrt{\theta}}
% \end{equation}

% Here we emphasize the time-related parameters only, as all other parameters are
% constants. We know that $ \lambda_\sigma \ge 0 $ as by definition of a density function, $ \phi(z) \ge 0 $, and because $ \sigma > 0 $ and  $ \theta \ge 0 $. Using our new results in (\ref{black_vol_prem}), we can now rewrite (\ref{black_qcall}) with only $ \mathbb{Q}\{F_T \ge K \mid \mathcal{F}_t) $ while incorporating $ \lambda_\sigma $ to arrive at
% \begin{equation} \label{black_vpcall}
%     C_{t, T}^F = D_{t, T}\left[\mathbb{Q}\{F_T \ge K \mid \mathcal{F}_t)(F - K) + F\int_{-\zeta}^{-\zeta + \sigma\sqrt{T - t}}\phi(z)dz\right]
% \end{equation}
% We have now expressed $ C_{t, T}^F $ as a sum of the difference $ F - K $ between the underlying and the strike weighted by the risk-neutral exercise probability $ \mathbb{Q}\{F_T \ge K) $ and the volatility premium weighted by $ F $.

% \section{Analyzing the vol premium}

% From our result in (\ref{black_vpcall}), it is implied that the value of European call options struck at the money is the discounted volatility premium weighted by $ F = K $. We make this explicit below by writing
% \begin{equation*}
%     C(K, t) = D_{t, T}K\int_{-\zeta'}^{-\zeta' + \sigma\sqrt{T - t}}\phi(z)dz = D_{t, T}K\int_{-\zeta'}^{\zeta'}\phi(z)dz, \text{where} -\zeta' = -\frac{1}{2}\sigma\sqrt{T - t}
% \end{equation*}
% However, since the greatest difference between the price of a European call option under the Black model and its intrinsic value is greatest at the money, one might be interested in how $ \lambda_\sigma $ changes as we vary the time $ t $ futures price $ F $. Because we already showed that the value of at the money Black call is completely determined by $ F $ and $ \lambda_\sigma $, our intuition tells us that a plot of $ \lambda_\sigma $ against $ F $ is most likely concave, with one maximum at $ F = K $. Taking the partial derivative $ \partial \lambda_\sigma / \partial F $, we thus have
% \begin{equation*}
%     \frac{\partial\lambda_\sigma}{\partial F} = \frac{\partial}{\partial F}\int_{-\zeta}^{-\zeta + \sigma\sqrt{\theta}}\phi(z)dz = \frac{\partial}{\partial F}\left[\Phi\left(-\zeta + \sigma\sqrt{\theta}\right) - \Phi(-\zeta)\right] = \phi\left(-\zeta + \sigma\sqrt{\theta}\right)\frac{1}{F\sigma\sqrt{\theta}} - \phi(-\zeta)\frac{1}{F\sigma\sqrt{\theta}}
% \end{equation*}
% Factoring the expression above and rearranging terms, we simply have
% \begin{equation} \label{black_vp_delta}
%     \frac{\partial\lambda_\sigma}{\partial F} = \frac{1}{F\sigma\sqrt{\theta}}\left[\phi\left(-\zeta + \sigma\sqrt{\theta}\right) - \phi(-\zeta)\right]
% \end{equation}
% We consider the cases $ F < K $, $ F = K $, and $ F > K $ to roughly understand how $ \lambda_\sigma $ changes as $ F $ changes. 

% \medskip

% $ F < K $: In this case, assuming $ \sigma\sqrt{\theta} > 0 $, $ \phi\left(-\zeta + \sigma\sqrt{\theta}\right) > \phi(-\zeta) $. We can thus conclude that
% \begin{equation*}
%     \left.\frac{\partial\lambda_\sigma}{\partial F} \middle|_{F < K}\right. > 0
% \end{equation*}
% $ F = K $: In this case, we see that $ -\zeta = -\zeta' = -\frac{1}{2}\sigma\sqrt{\theta}$, so $ -\zeta' + \sigma\sqrt{\theta} = \zeta' $. Due to the symmetry of $ \phi(z) $ around 0, we have that $ \phi(\zeta') - \phi(-\zeta') = 0 $, so we conclude that
% \begin{equation*}
%     \left.\frac{\partial\lambda_\sigma}{\partial F} \middle\vert_{F = K}\right. = 0
% \end{equation*}
% $ F > K $: In this case, assuming $ \sigma\sqrt{\theta} > 0 $, $ \phi\left(-\zeta + \sigma\sqrt{\theta}\right) < \phi(-\zeta) $. We can thus conclude that
% \begin{equation*}
%     \left.\frac{\partial\lambda_\sigma}{\partial F} \middle\vert_{F > K}\right. < 0
% \end{equation*}
% This confirms our intuition that $ \lambda_\sigma $ is greatest when $ F = K $. Plots of $ \lambda_\sigma $ over $ F $ and $ \partial \lambda_\sigma / \partial F $ over $ F $ in Figures \ref{vol_prem} and \ref{vol_prem_delta} show more complex dynamics, but still confirm our results. Interestingly, because of the lognormal dynamics assumed in the Black model, the plots of $ \lambda_\sigma $ and $ \partial\lambda_\sigma / \partial F $ over $ F $ are positively skewed.

% \section{Conclusion}

% In this article, we derived the Black formula by means of conditional expectation, which allowed us to re-express the formula in terms of risk-neutral exercise probabilities and isolate the volatility premium embedded in an option's price. We were able to avoid constructing the risk-free portfolio and solving the resulting PDE as outlined in the original Black-Scholes (1973) and Black (1976) papers. The probabilistic approach may be preferred for its financial intuition over the PDE approach, which requires a change of variable into dimensionless groups, thus temporarily causing the equation to lose all financial meaning.

% \pagebreak

% \section{Figures}

% All figures were plotted in R.

% \begin{figure}[h!]
%     \centering
%     \includegraphics[scale = 0.6]{images/black_vol_prem_crop.png}
%     \caption{Plot of $ \lambda_\sigma $ over $ F $ with arbitrary parameters $ \sigma = 0.2 $, $ K = 50 $, $ T - t = 1 $.}
%     \label{vol_prem}
% \end{figure}
% \begin{figure}[h!]
%     \centering
%     \includegraphics[scale = 0.6]{images/black_vol_prem_delta_crop.png}
%     \caption{Plot of $ \partial\lambda_\sigma / \partial F $ over $ F $ with arbitrary parameters $ \sigma = 0.2 $, $ K = 50 $, $ T - t = 1 $.}
%     \label{vol_prem_delta}
% \end{figure}

\end{document}