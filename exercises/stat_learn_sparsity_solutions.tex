\documentclass{article}
\usepackage[margin=1in]{geometry}
\usepackage{amsmath, amssymb, amsfonts, enumitem, fancyhdr, tikz}
% get rid of paragraph indent
\setlength{\parindent}{0 pt}
% allow section.equation numbering
\numberwithin{equation}{section}
\usepackage{hyperref}
% make the link colors blue, as well as cite colors. urls are magenta
\hypersetup{colorlinks, linkcolor=blue, citecolor=blue, urlcolor=magenta}

\title{Statistical Learning with Sparsity \\ \Large Exercise Solutions}
\author{Derek Huang\thanks{SMBC Capital Markets, Quantitative Strategists.}}
\date{Match 13, 2022}

\begin{document}

\newcommand{\newtocsubsection}[1]{%
    \subsection*{#1} \addcontentsline{toc}{subsection}{#1}%
}

\maketitle

%\newpage

\tableofcontents

\newpage

\section{Introduction}

During my time as an undergraduate student at NYU, there was a period where
I found it interesting how the theory of convex optimization was applied by
the field of supervised learning to fit models. One of the topics I was
interested in was sparse optimization, mostly the kinds where sparsity was
enforced by a convex norm constraint, ex. $ \ell^1 $ norm for vectors,
nuclear norm for matrices.

\medskip

To satisfy my interests, I ended up purchasing Hastie, Tibshirani, and
Wainwright's \textit{%
    Statistical Learning with Sparsity: The Lasso and Generalizations%
}, seeing that it was a monograph on the topic. Despite enjoying the portions
of the book that I found time to read, I was irrationally afraid of
attempting the exercises, mostly because of a subconscious belief that failure
to satisfactorily complete meaningfully challenging exercises would somehow
invalidate both my intelligence and my worth as an individual. I no longer
feel the same way, although I still think that in general, my own intelligence
and individual worth are nothing significant.

\medskip

Regardless, the exercise solutions in this document exist to be a testament
to my own efforts towards understanding and will hopefully be a resource to
anyone else who may be attempting these exercises.

\section{The Lasso for Linear Models}

\newtocsubsection{Exercise 2.1}

Let $ \mathbf{x}_j  \in \mathbb{R}^N $ denote the $ j $-th column of the
predictor matrix $ \mathbf{X} \in \mathbb{R}^{N \times d} $, and let
$ \mathbf{y} \in \mathbb{R}^N $ denote the response vector as usual. Under
the assumptions of section 2.4, we have $ \frac{1}{N}\mathbf{1}^\top
\mathbf{y} = 0 $, $ \frac{1}{N}\mathbf{1}^\top\mathbf{x}_j = 0 $,
$ \frac{1}{N}\mathbf{x}_j^\top\mathbf{x}_j = 1 $\footnote{
    In words, the predictors are standardized to zero mean and unit variance,
    while the response is standardized to zero mean.
}.

\medskip

The lasso problem, as introduced in the book, can be written as
\begin{equation} \label{eq:standard_lasso}
    \min_\mathbf{w}\frac{1}{2N}\Vert\mathbf{y} - \mathbf{Xw}\Vert_2^2 +
    \lambda\Vert\mathbf{w}\Vert_1
\end{equation}

Let us replace $ \lambda \in (0, \infty) $ with $ \lambda_j \in
(0, \infty) $, where $ \forall j \in \{1, \ldots d\} $, $ \lambda_j
\Rightarrow w_{j'} \in \{1, \ldots d\}, j' \ne j, w_{j'} = 0 $.
With $ \lambda \triangleq \lambda_j $, only $ w_j $ may be nonzero, and
so in that case we can rewrite (\ref{eq:standard_lasso}) where
$ \lambda \triangleq \lambda_j $ as
\begin{equation} \label{eq:single_lasso}
    \min_{w_j}\frac{1}{2N}\sum_{k = 1}^N(y_k - x_{kj}w_j)^2 +
    \lambda_j|w_j|
\end{equation}

Taking the subderivative of (\ref{eq:single_lasso}) at its minimizer
$ \hat{w}_j $, we have
\begin{equation*}
    0 \in \left\{
        -\frac{1}{N}\sum_{k = 1}^Nx_{kj}(y_k - x_{kj}\hat{w}_j)
    \right\} +
    \lambda_j\partial|\hat{w}_j| =
    \left\{-\frac{1}{N}\mathbf{x}_j^\top\mathbf{y} + \hat{w}_j\right\} +
    \lambda_j\partial|\hat{w}_j|
\end{equation*}

Note that $ \hat{w}_j = 0 \Rightarrow \partial|\hat{w}_j| = [-1, 1] $. We
must then have $ \left|\frac{1}{N}\mathbf{x}_j^\top\mathbf{y}\right| \le
\lambda_j \Rightarrow \left|\frac{1}{N}\mathbf{x}_j^\top\mathbf{y}\right| $
is the smallest value of $ \lambda $ that guarantees $ \hat{w}_j = 0 $
conditional on $ \hat{w}_{j'} = 0 $, $ \forall j' \in \{1, \ldots d\} $,
$ j' \ne j $. Therefore, the smallest value of $ \lambda $,
which we denote $ \tilde{\lambda} $, that unconditionally guarantees
$ \hat{\mathbf{w}} = \mathbf{0} $, is such that
\begin{equation*}
    \tilde{\lambda} \triangleq \frac{1}{N}\max\big\{\big|\mathbf{x}_1^\top\mathbf{y}\big|, \ldots \big|\mathbf{x}_d^\top\mathbf{y}\big|\big\}
\end{equation*}


\newtocsubsection{Exercise 2.2}

TODO


\end{document}