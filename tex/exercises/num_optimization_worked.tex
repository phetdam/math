\documentclass{article}

% standard article setup for worked exercises
% Standard \documentclass{article} preamble setup used for worked exercises.
%
% This should be specified with \input right after the \documentclass{article}
% command is issued to all the necessary configuration as follows:
%
%   % standard article setup for worked exercises
%   % Standard \documentclass{article} preamble setup used for worked exercises.
%
% This should be specified with \input right after the \documentclass{article}
% command is issued to all the necessary configuration as follows:
%
%   % standard article setup for worked exercises
%   % Standard \documentclass{article} preamble setup used for worked exercises.
%
% This should be specified with \input right after the \documentclass{article}
% command is issued to all the necessary configuration as follows:
%
%   % standard article setup for worked exercises
%   \input{../utils/exercises_preamble}
%
% The path used might change depending on directory structure refactoring.

\usepackage[margin=1in]{geometry}
\usepackage{amsmath, amssymb, amsfonts, enumitem, fancyhdr, tikz}
% ensure that bibliography is part of toc, is numbered, and doesn't also
% include the table of contents as a link within itself
\usepackage[nottoc, numbib]{tocbibind}
% get rid of paragraph indent
\setlength{\parindent}{0 pt}
% allow section.equation numbering
\numberwithin{equation}{section}
\usepackage{hyperref}
% make the link colors blue, as well as cite colors. urls are magenta
\hypersetup{colorlinks, linkcolor=blue, citecolor=blue, urlcolor=magenta}

%
% The path used might change depending on directory structure refactoring.

\usepackage[margin=1in]{geometry}
\usepackage{amsmath, amssymb, amsfonts, enumitem, fancyhdr, tikz}
% ensure that bibliography is part of toc, is numbered, and doesn't also
% include the table of contents as a link within itself
\usepackage[nottoc, numbib]{tocbibind}
% get rid of paragraph indent
\setlength{\parindent}{0 pt}
% allow section.equation numbering
\numberwithin{equation}{section}
\usepackage{hyperref}
% make the link colors blue, as well as cite colors. urls are magenta
\hypersetup{colorlinks, linkcolor=blue, citecolor=blue, urlcolor=magenta}

%
% The path used might change depending on directory structure refactoring.

\usepackage[margin=1in]{geometry}
\usepackage{amsmath, amssymb, amsfonts, enumitem, fancyhdr, tikz}
% ensure that bibliography is part of toc, is numbered, and doesn't also
% include the table of contents as a link within itself
\usepackage[nottoc, numbib]{tocbibind}
% get rid of paragraph indent
\setlength{\parindent}{0 pt}
% allow section.equation numbering
\numberwithin{equation}{section}
\usepackage{hyperref}
% make the link colors blue, as well as cite colors. urls are magenta
\hypersetup{colorlinks, linkcolor=blue, citecolor=blue, urlcolor=magenta}


\title{Numerical Optimization \\ \Large Worked Exercises}
\author{%
    Derek Huang\thanks{TD Securities, Quantitative Modeling and Analytics.}
}
\date{July 9, 2022}

\begin{document}

% define the \newtocsubection command
% Adds the \newtocsubsection command.
%
% Useful for making sure that subsections are also including the table of
% contents produced by LaTeX using the \tableofcontents command.
%
% Standard usage in some LaTeX file is as follows:
%
%   % define the \newtocsubection command
%   % Adds the \newtocsubsection command.
%
% Useful for making sure that subsections are also including the table of
% contents produced by LaTeX using the \tableofcontents command.
%
% Standard usage in some LaTeX file is as follows:
%
%   % define the \newtocsubection command
%   % Adds the \newtocsubsection command.
%
% Useful for making sure that subsections are also including the table of
% contents produced by LaTeX using the \tableofcontents command.
%
% Standard usage in some LaTeX file is as follows:
%
%   % define the \newtocsubection command
%   \input{../utils/newtocsubsection}

%%
% Start unnumbered subsection, also adding it to the table ofcontents.
%
% Arguments:
%   #1  subsection title
%
\newcommand{\newtocsubsection}[1]{%
    \subsection*{#1} \addcontentsline{toc}{subsection}{#1}%
}


%%
% Start unnumbered subsection, also adding it to the table ofcontents.
%
% Arguments:
%   #1  subsection title
%
\newcommand{\newtocsubsection}[1]{%
    \subsection*{#1} \addcontentsline{toc}{subsection}{#1}%
}


%%
% Start unnumbered subsection, also adding it to the table ofcontents.
%
% Arguments:
%   #1  subsection title
%
\newcommand{\newtocsubsection}[1]{%
    \subsection*{#1} \addcontentsline{toc}{subsection}{#1}%
}


\maketitle

%\newpage

\tableofcontents

\newpage

\section{Introduction}

After developing an interest in optimization, I started looking for good
textbooks I could feed my interest with and learn from. One of which was
Nocedal and Wright's \textit{Numerical Optimization}, which I accidentally
stumbled upon while reading the
\href{%
    https://docs.scipy.org/doc/scipy/reference/generated/%
    scipy.optimize.minimize.html%
}{documentation for \texttt{scipy.optimize.minimize}}, as I noticed that
several\footnote{%
    As of this writing, eight: BFGS, Newton-CG, dogleg, trust-ncg,
    trust-krylov, trust-exact.%
}
of the implemented minimization algorithms were from Nocedal and Wright's
book. That prompted my interest and led me to believe that Nocedal and
Wright's book would be very helpful for learning about and implementing the
algorithms that people use in production settings. I already had Boyd and
Vandenberghe's \textit{Convex Optimization} at the time, and although it is
excellent for understanding the theory, it was also only focused on convex
problems and most importantly, had very little discussion of real-life
algorithms. Therefore, getting the book was a no-brainer for me, given my
interests.

\medskip

% standardized closing transition
% Standard blurb inserted as the closing to an introduction.
%
% So far, the TeX files in exercises that have my worked solutions to various
% textbook exercises all have introductions explaining why I picked up the
% book and so on. This file is used to standardize the closing transition
% paragraph used before starting the actual exercise sections.
%
% Typical usage is by using \input as follows:
%
%   % standardized closing transition
%   % Standard blurb inserted as the closing to an introduction.
%
% So far, the TeX files in exercises that have my worked solutions to various
% textbook exercises all have introductions explaining why I picked up the
% book and so on. This file is used to standardize the closing transition
% paragraph used before starting the actual exercise sections.
%
% Typical usage is by using \input as follows:
%
%   % standardized closing transition
%   % Standard blurb inserted as the closing to an introduction.
%
% So far, the TeX files in exercises that have my worked solutions to various
% textbook exercises all have introductions explaining why I picked up the
% book and so on. This file is used to standardize the closing transition
% paragraph used before starting the actual exercise sections.
%
% Typical usage is by using \input as follows:
%
%   % standardized closing transition
%   \input{../utils/intro_close}
%
% The path used might change depending on directory structure refactoring.

Regardless, the worked exercises in this document exist as a testament to my
own efforts towards understanding and will hopefully be a resource to
anyone else who may be attempting these exercises.

\medskip

\fbox{%
    \parbox{\textwidth}{%
        \textbf{Disclaimer:} This is \textcolor{red}{\textbf{not}} an official
        solution guide for the text. I \textbf{strongly} \textbf{recommend}
        that one attempt the text's exercises \textbf{on} \textbf{their}
        \textbf{own} before consulting this document, as I believe active
        self-learning truly plays an outsized role in determining the depth of
        one's understanding. One's instructors are there to guide and support,
        but all must walk their paths to understanding themselves.%
    }%
}

%
% The path used might change depending on directory structure refactoring.

Regardless, the worked exercises in this document exist as a testament to my
own efforts towards understanding and will hopefully be a resource to
anyone else who may be attempting these exercises.

\medskip

\fbox{%
    \parbox{\textwidth}{%
        \textbf{Disclaimer:} This is \textcolor{red}{\textbf{not}} an official
        solution guide for the text. I \textbf{strongly} \textbf{recommend}
        that one attempt the text's exercises \textbf{on} \textbf{their}
        \textbf{own} before consulting this document, as I believe active
        self-learning truly plays an outsized role in determining the depth of
        one's understanding. One's instructors are there to guide and support,
        but all must walk their paths to understanding themselves.%
    }%
}

%
% The path used might change depending on directory structure refactoring.

Regardless, the worked exercises in this document exist as a testament to my
own efforts towards understanding and will hopefully be a resource to
anyone else who may be attempting these exercises.

\medskip

\fbox{%
    \parbox{\textwidth}{%
        \textbf{Disclaimer:} This is \textcolor{red}{\textbf{not}} an official
        solution guide for the text. I \textbf{strongly} \textbf{recommend}
        that one attempt the text's exercises \textbf{on} \textbf{their}
        \textbf{own} before consulting this document, as I believe active
        self-learning truly plays an outsized role in determining the depth of
        one's understanding. One's instructors are there to guide and support,
        but all must walk their paths to understanding themselves.%
    }%
}


\section{Line Search Methods}

\newtocsubsection{Exercise 3.1}

Although not required by the problem, we use a generic line search function
implemented via templates in C++. The type of line search method used is
simply specified via a direction search functor, which returns the search
direction from the current guess, and a step search functor, which returns the
step size from the current guess. For this exercise, we use a backtracking
line search implementing step search functor.

\medskip

TODO

\newtocsubsection{Exercise 3.3}

Let us define $ f : \mathbb{R}^d \rightarrow \mathbb{R} $ s.t. for
$ \mathbf{Q} \succeq m\mathbf{I} \in \mathbb{R}^{d \times d} $,
$ m \in (0, \infty) $, $ \mathbf{b} \in \mathbb{R}^d $,
\begin{equation} \label{eq:3.3.1}
    f(\mathbf{x}) \triangleq
    \frac{1}{2}\mathbf{x}^\top\mathbf{Qx} - \mathbf{b}^\top\mathbf{x}
\end{equation}

Note that $ \mathbf{Q} \succeq m\mathbf{I} \Rightarrow f $ is strongly convex.
Let us also define
$ f_{\mathbf{x}, \mathbf{p}} : \mathbb{R} \rightarrow \mathbb{R} $ for
$ \mathbf{x}, \mathbf{p} \in \mathbb{R}^d $ s.t.
\begin{equation} \label{eq:3.3.2}
    f_{\mathbf{x}, \mathbf{p}}(\alpha) \triangleq
    f(\mathbf{x} + \alpha\mathbf{p})
\end{equation}

Here $ \mathbf{p} \in \mathbb{R}^d $ is implicitly assumed to be a descent
direction, i.e. $ \mathbf{p}^\top\nabla f(\mathbf{x}) < 0 $, which is a fact
that will prove later to be quite important.

\medskip

TODO

\end{document}
